\documentclass{eplDoc}
\usepackage{placeins}


\newcommand{\docType}	{Assignment 3 : Semantic word clustering}
\newcommand{\docDate}	{03/05/2012}
\newcommand{\docAuthor}	{gr10 : Mulders Corentin, Pelsser Francois}
\newcommand{\courseCode}{LINGI2263}
\newcommand{\courseName}{Computational linguistics}
\usepackage{syntax}
\begin{document}
\maketitle
\newpage

\section{Global architecture}

Our program is composed of three parts : 
\begin{enumerate}
	\item Lemmatization and part of speech tagging (tag.py)
	\item Filtering of stop words and stop part-of-speech tags (filter.py)
	\item Clustering  %TODO py name
\end{enumerate}

The first two parts do the preprocessing on the text files. The first one takes a definitions file as input and produces a file with words in the definitions tagged with a part-of-speech tag and lemmatized. The second takes a file with tagged definitions and removes words that match either a stop word or a stop tag. \\ 

The third part is the main part, it takes a preprocessed file as input and produces an output file with the culsters of words. 

\section{Discussion on the results}

\subsection{Results depending on the value of K}

\subsection{How to automate the evaluation of the quality of the clustering} %and ressources needed

\subsection{Possible ways to enhance our results} % and reason for error sin our results


\end{document}
