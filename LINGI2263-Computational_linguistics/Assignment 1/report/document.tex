\documentclass{eplDoc}



\newcommand{\docType}	{Assignment 1 : Corpus Processing}
\newcommand{\docDate}	{01/03/2012}
\newcommand{\docAuthor}	{gr10 : Mulders Corentin, Pelsser François}
\newcommand{\courseCode}{LINGI2263}
\newcommand{\courseName}{Computational linguistics}
\usepackage{syntax}
\begin{document}
\maketitle
\newpage

\section{Global acrhitecture}

\subsection{Program architecture}

Our program is composed of three parts. The first and main part is a set of unitex graphs that are used to single out the desired data with tags. The second part is a python script that applies these graphs to the files to get tagged files. The last one parses the tagged files to create a cleaned up results file. 

\subsection{Use of Unitex tools}

We use unitex to perform the patterns matchings that allow us to find the informations in the text. What we did is that for each desired information (such as gender, age, height,...) we created a unitex graph file. This graph can be used to ad tags around the wanted values. For example the age graph would surround all occurences of a patients age with \{age\} and \{/age\} in the text file. \\ 
Some of these graphs are very basic like the gender graph that simply looks for occurrences of "male" or "female". Others are more complex like the height graph that needs to be able to transform a height in the form "5 feet 3 inches" to "5'3"" for example, as well as specify in the tag what the unit used is so that our python script can convert it later. 

\section{Extraction patterns}

\subsection{Context and other idiosyncrasies used in solution}

\subsection{Advantages and shortcomings of our designs choices}

\subsection{Possible ways to improve results}

\end{document}
