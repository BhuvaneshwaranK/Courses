\documentclass{eplDoc}



\newcommand{\docType}	{Définition de la syntaxe}
\newcommand{\docDate}	{17/02/2012}
\newcommand{\docAuthor}	{Mulders Corentin, Forêt Nicolas, Pelsser François}
\newcommand{\courseCode}{LINGI2132}
\newcommand{\courseName}{Langages et traducteurs}
\usepackage{syntax}
\begin{document}
\maketitle
\newpage

\section{Introduction}
Our language is an oriented oriented language that uses message passing.  
Everything element is an object.  Basis classes are : Integer, Boolean, Tuple, 
Stdio and Messages.  To communicate with an object, we send it messages the 
object can then either manage the message or sent it to an other object.
It is dynamically typed. %TODO typed?? not sure and increase the sentence quality 
\section{Syntax description}

\begin{grammar}
<identifier> ::= `a..z' \{ `a..z,A..Z,\_,0..9,'\} 

<class\_identifier> ::= `A..Z' \{ `a..z,A..Z,\_,0..9,'\} 

<tuple> ::= `['`]' \alt `[' <element> \{`,' <element>\}`]'

<message> ::= `{'<identifier> \{`,' <element>\}`}'

<message\_sending> ::= <element> `<-' <message>

<boolean\_value> ::= true | false

<element> ::= <identifier> 
\alt<message\_sending>
\alt<instantiation>
\alt \{<digits>\}
\alt <boolean\_value>
\alt <operation>
\alt <message>
\alt <tuple>
\alt \# | self | super
\alt <element>.<identifier> 

<operation> ::= <element> <binary\_operator> <element>
\alt <unary\_operator> <element>
\alt (<operation>)

<binary\_operator> ::= `+' | `-' | `*' | `/' | `\%' 
\alt `<=' | `=>' | `<' | `>' | `==' | `!=' 
\alt `and' | `or'

<unary\_operator> ::= ! | - 

<instantiation> ::= <class\_identifier> `<-' `{new' \{`,' <element>\}`}'


<assignment\_expression> ::= <identifier> `=' <element>

<statement> ::= <message\_sending> `;'
\alt <assignment\_expression> `;'
\alt return <element> `;'
\alt <while\_statement>
\alt <if\_statement>

<comment> ::= // .. suite de caracteres ASCII .. \textbackslash n

<block code> ::= { <statement> } { <comment> } 

<while\_statement> ::= `while' `(' <expression> `)'  `{'<block code>`}'

<if\_statement> ::= `if' `(' <expression> `)'  `{'<block code>`}'




<class> ::= `class' <class\_identifier> (extends <identifier>)? `{' <class body> `}'

<class\_body> ::= \{<attribute\_declaration>\} \{<message\_declaration>\}

<message\_declaration> ::= `def'  `{'<identifier> \{`,' <identifier>\}`}' `\{'<code\_block>`\}'

<attribute\_declaration> ::= <identifier> `;'

\end{grammar}


\section{Data types}

We would define some basic classes to store data. 

\begin{itemize}
	\item \textbf{Integer} represents integer numbers. 
	\item \textbf{Boolean} represents a $<$boolean\_value$>$.
	\item \textbf{Tuple} represents a collection (a $<$tuple$>$ corresponds to a Tuple object).
	\item \textbf{Message} is a subclass of Tuple and represents a $<$message$>$.
	\item \textbf{Stdio} represents the standard input and output. 
\end{itemize}

\section{Syntaxic sugars}
\label{synt}
\{digits\} is translated in Integer$\leftarrow$(new, \{digits\}) \\
4 + 2 is translated in Integer$\leftarrow$(new, 4)$\leftarrow$(add, Integer$\leftarrow$(new, 2)) \\
The operator 'not' is a unary operator. Example: Integer$\leftarrow$(new, 2)$\leftarrow$leq(4)$\leftarrow$(not). Syntactic sugar: !(2 $<=$ 4)\\

Binary operators : 
\begin{array}{|c|c|}
\hline
+ & sum \\
- & difference \\
* & multiplication \\
/ & division \\
\% & mod \\
and & and \\
or & or \\
<= & leq \\
>= & geq \\
== & eq \\
!= & neq \\
< & lt \\
> & gt \\
\hline
\end{array}
Unary operators : 
\begin{array}{|c|c|}
\hline
! & not \\
- & minus \\
\hline
\end{array}

\section{Semantics}
As there is no primitive types, we define a notation $<$Class value$>$ which represent an instance of the class "Class" wich has the value "value"
\subsection{Basic classes}
\subsubsection{Integer}
The class integer represents all integer.\\
Sequences of digits are considered as an instance of the Integer class :\\
eval(n) = $<$Integer n$>$ Where n $\in \mathbb\{Z\}$\\
Integer class respond to these messages :\\
sum, difference, multiplication, division, mod, minus, leq, geq, eq, neq, lt, gt\\
The first part (sum, difference, multiplication, division, mod, minus), return a new Integer that has as value the result of the operation\\
Ex: eval($<$Integer 4$>$ $<$- \{sum, $<$Integer 2$>$\}) = $<$Integer 6$>$\\
The second part (leq, geq, eq, neq, lt, gt), return a new Boolean that also has as value the result of the operation\\
Ex: eval($<$Integer 4$>$ $<$- \{lt, $<$Integer 2$>$\}) = $<$Boolean true$>$

\subsubsection{Boolean}
This class represents the boolean values true and false.\\
The literals true and false are considered as instance of the Boolean class:
eval(true) = <Boolean true>\\
eval(false) = <Boolean false>\\
Boolean class respond to these messages by creating a new boolean instance with as value thhe result of the operation:\\
not, and, or
Ex : eval($<$Boolean true$>$ $<$- \{and $<$Boolean false$>$\}) = $<$Boolean false$>$

\subsubsection{Tuple}
The class tuple is used to keep collection of items.
Elements separated by comas and surrounded by square brackets are considered as a tuple.
eval([$<$Integer 4$>$,$<$Integer 3$>$,$<$Boolean true$>$]) = $<$Tuple $<$Integer 4$>$,$<$Integer 3$>$,$<$Boolean true$>$$>$
Tuples responses to messages :
\begin{description}
    \item[add, item] a dd item at the end of the tuple
    \item[remove, i] remove the item at position i of the tuple
    \item[itemAt, i] return the item at position i of the tuple
\end{description}
The position of the items are Integer instances incrementing of 1 for every item and beginning at 0.\\



\subsubsection{Message}
The class Message represents messages sends to objects.\\
Elements separated by comas and surrounded by curly brackets are considered as a message.\\
The first element of the message is an identifier.  It has to be formatted as an identifier (see BNF).\\
eval(\{messageName, param1, param2\}) = $<$Message messageName,param1,param2$>$


\subsubsection{Stdio}
This class is used to print elements on an output.  There is a global constant "stdio" wich represent the standard output.  \\
This class manages the \{print element\} message that print a representation of the element on the output.  



\subsection{Operations and assignments}
Let E represent the environment. \\ 
By default eval(expression) = F where F is a new Boolean in E and F as false as value. Else the value of the evaluations of expressions is defined by the following rules : 
\begin{itemize}
\item eval(n, E) = i where i is a new Integer in E and i has n as value if n $\in \mathbb{Z}$ 

\item eval(b, E) = B where B is a new Boolean in E and B has b as value if b $\in$ \{true, false\} 

\item eval(id, E) = O where O is the object in E referenced by the identifier id if id is a valid identifier.

\item eval(id1.id2, E) = O where O is the Object referenced by id2 from the instance variable of the object referenced by id1 in E if id1 is a valid identifier and id2 is a valid identifier for a variable from the object of id1. 

\item eval(\{t1, t2, ... , tn\}, E) = T where T is a new tuple in E that contains eval(t1), eval(t2), ... , eval(tn) if eval(ti) is an Object $\forall i \in [1,n]$. 

\item eval(\{id, t1, ... , tn\}, E) = M where M is a new message in E with id as identifier and that contains eval(t1), eval(t2), ... , eval(tn) if eval(ti) is an Object $\forall i \in [1,n]$. 

\item eval(n1 + n2, E) = eval(n1) $<$-\{sum, eval(n2)\}) if eval(n1) is an Object and eval(n2) is an Object. \\
\begin{center}
\vdots  \\
\end{center}
 
The same is valid for syntaxic sugars for the other binary operators with the messages from section \ref{synt}.                      
\begin{center}
\vdots \\
\end{center}
  
\item eval(!n1, E) = eval(n1 $<$-\{not\}) if eval(n1) is an Object. 
\item eval(-n1, E) = eval(n1 $<$-\{minus\}) if eval(n1) is an Object. 

\end{itemize}



\subsection{Class definition}
A class contains 4 informations : 
\begin{itemize}
    \item a name
    \item a super class
    \item attributes 
    \item messages definitions
\end{itemize}

There are 2 predefined messages, "new" and "default".  The first one is called 
to create an instance of a class.  The number of arguments depends of the 
definition of the new message(s) in the class.  The second, "default", is called
when the object received a message wich is not in its managed messages or in 
managed messages of one of its parents, this method can get the message to send 
it to another object with the character "\#". 


Every class attribute of the class have to be declared at the beginning of this one.  
They cannot be instanced during the declaration.

To define the managed message we should define every of them.  This take place after the attributes declaration.  For every message we have to set the name as an identifier, then add an identifier for every parameter. The parameter's identifier can then be used to retriveve parameter's values. 
The body can contain return statement that will define what will be returned.
Ex : 
\begin{lstlisting}
    def {messageName, param1, param2}
    {
        return param1 + param2;
    }
\end{lstlisting}
In this exemple we have a message named "messageName", it takes 2 parameters that can be used in the body of the message definition with identifier "param1" and "param2".  The message call return the sum of the 2 parameters.


\subsection{Sending Messages}

\begin{itemize}
	\item instr(o$<$-message, E) = instr(block code, E$\cup$ \{self$\rightarrow$ eval(o, E), \#$\rightarrow$ eval(message, E)) where block code is found as defined below if o is the identifier of an Object in E. 
\end{itemize}

\subsubsection{Block code executed when a message is sent}
\begin{itemize}
	\item If the identifier of the message corresponds to a message defined for the receiving object then the block code defined for this object is executed.
	\item If none of the above, if the identifier of the message corresponds to a message defined for the receiving object's super class then the block code defined for this object's super class is executed.
	\item If none of the above, if the identifier of the message corresponds to a message defined for the receiving object's default receiving class then the block code defined for this object's default class is executed.
	\item If none of the above, if the identifier of the message corresponds to a message defined for the default receiving class of the the receiving object's super class then the block code defined for the default class of this object's super class is executed.
	\item If none of the above, no code is executed. 
\end{itemize}

To summarize, the message is sent first to the destination object, if it can't do anything with it it tries it's super class messages, if no super class in the hierarchy knows this message then we try to send it to the destination object's default receiver (for example a method that prints an error message), if this receiver isn't defined we try to use one from the super class. 

\subsection{Control Sequences} 
\subsubsection{Conditional execution}
\begin{itemize}
	\item instr(if(cond) \{block code, E\}) = instr(block code, E) if eval(cond, E \}) = $<$Boolean true$>$
	\item instr(if(cond) \{block code 1, E\} else \{block code 2, E\}) = instr(block code 1, E) if eval(cond, E) = $<$Boolean true$>$  
  \item instr(if(cond) \{block code 1, E\} else \{block code 2, E\}) = instr(block code 2, E) if eval(cond, E) = $<$Boolean false$>$           
                       
\end{itemize}

\subsubsection{While loop}

\begin{itemize}
	\item instr(while(cond) \{block code, E\}) = instr(block code, E) while eval(cond, E) = $<$Boolean true$>$
     
                       
\end{itemize}

\section{Code examples}

\subsection{Definition of a new class}
\begin{lstlisting}
class Point{
		// two instance variables x and y declared
    x;
    y;

		// what to do when an instance is created
    def {new, px, py}
    {
        x = px;
        y = py;
    }
		
		//definition of an add message
    def {add, px, py}
    {
        x = x + px;
        y = y + py;
    }

		// getters
    def {getX}
    {
        return x;    
    }
    
    def {getY}
    {
        return y;    
    }
		
		
		
    def {gt, p2}
    {
        px = p2 <- getX;    
        py = p2 <- getY;    
        return (x*x + y*y) > (px*px + py*py)
     }
}

p = Point <- {new, 12, 14}
p2 = Point <- {new,13 ,28} + p

stdout <- {print, p > p2}

\end{lstlisting}

\end{document}
