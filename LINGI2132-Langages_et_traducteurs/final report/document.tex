\documentclass{eplDoc}



\newcommand{\docType}	{Interpreter for Postal}
\newcommand{\docDate}	{18/05/2012}
\newcommand{\docAuthor}	{gr10 : Mulders Corentin, Foret Nicolas, Pelsser Francois}
\newcommand{\courseCode}{LINGI2132}
\newcommand{\courseName}{Langages et traducteurs}
\usepackage{syntax}
\begin{document}
\maketitle
\newpage

\section{LL1 Grammar}
The first thing to do is to make our grammar LL1. \\
The initial grammar is in grammar.txt\\
The first change we made is removing the lambda production from block code because we don't want the empty string to be derivable.  This was initially the only we thought we would have to add to turn the grammar LL1.
\\
Then the major problems were loops in the rules. For instance the elements resulting of the sending of a message to another element can be used too to be the recipient of a message. \\
To handle that we worked with prefixes and suffixes.\\
The others modifications are due to previous mistakes.\\
The resulting grammar is in gLL1.txt\\
This grammar is not totally LL1.  When we check it we receive errors like:
\begin{lstlisting}
Conflit LL(1) entre les deux regles 
<element suffixe> --> <operation suffixe> <element suffixe> 
<element suffixe> --> <lambda>
la chaine problematique est ">".
\end{lstlisting}
Which we didn't really understand because <lambda> is empty and can't then begin by ">" and nothing else can begin by that but <operation sffixe>.\\
We then decided to generate the table and we checked that it was correct and that was. So we left it that way.


\section{Parsing}

For parsing we used a recursive approach : 
the parser is located in the sources in postal/parser/LL1Parser.java

\section{Transforming syntax tree}
The next step is to transform the syntax tree to an abstract syntax tree that can be executed.
This is done by using the automatic skeleton generation from GTools and then filling the blanks.\\
We used a class to improve the result by associating non-terminals with objects types.
The final code is in postal/parser/ST2AST.java

\section{Other important classes}
The launcher class is Postal.java, it get the source then execute it with postal/parser/Executer.java\\
In postal/classes and postal/objects are the classes and objects representic basic classes and objects from our langage and user defined classes and object.\\
In postal/ast are all the nodes to build the ast.

\section{Testing}
We tested some easy and complex programs.  We built a jar which take an argument at execution and acts as the full interreter for our language.  If the argument is a path to a file the content of the file is executed, else the argument is directly executed as program.\\
We also created some wrong sources files to see how errors are managed.
\subsection{correct sources}
All tests are available in the tests directory and the sorting algorithm is also there (tri.po).\\
Here are some interesting tests.
\subsubsection{t1.po operations}
\begin{lstlisting}
    
stdio<-{print,(4/2)+4};
stdio<-{print,(4/2)+14};
\end{lstlisting}
result:
\begin{lstlisting}
    
PostalSTDIO : [(Integer Object) : 6]
PostalSTDIO : [(Integer Object) : 16]
\end{lstlisting}

\subsubsection{t4.po loop}
\begin{lstlisting}
    
a=10;
while(a>1)
{
    stdio<-{print,a};
    a=a-1;
}
\end{lstlisting}
result:
\begin{lstlisting}
    
PostalSTDIO : [(Integer Object) : 10]
PostalSTDIO : [(Integer Object) : 9]
PostalSTDIO : [(Integer Object) : 8]
PostalSTDIO : [(Integer Object) : 7]
PostalSTDIO : [(Integer Object) : 6]
PostalSTDIO : [(Integer Object) : 5]
PostalSTDIO : [(Integer Object) : 4]
PostalSTDIO : [(Integer Object) : 3]
PostalSTDIO : [(Integer Object) : 2]
\end{lstlisting}
\subsubsection{t9.po recursivity}
\begin{lstlisting}
    
class Counter{
    {
        n;
    }
    {
        def{count, i}
        {
            stdio<-{print,i};
            if(i<self.n){self<-{count, i+1};}
        }
        def{set, i}
        {
            self.n=i;
        }
            
    }
}

c=Counter<-{new};
c<-{set, 5};
c<-{count,0};
\end{lstlisting}
result:
\begin{lstlisting}
    
PostalSTDIO : [(Integer Object) : 0]
PostalSTDIO : [(Integer Object) : 1]
PostalSTDIO : [(Integer Object) : 2]
PostalSTDIO : [(Integer Object) : 3]
PostalSTDIO : [(Integer Object) : 4]
PostalSTDIO : [(Integer Object) : 5]
\end{lstlisting}

\subsubsection{t11.po inherirance}
\begin{lstlisting}
class A 
{
    {
        a;    
    }
    {
        def{showa}
        {
            stdio<-{print,6};
        }       
    }
}
class B extends A
{
    {
        b;
    }
    {
        def{showb}
        {
            stdio<-{print,4};
        }       
    }
}
ob = B<-{new};
ob<-{showb};
ob<-{showa};
    
\end{lstlisting}
result:
\begin{lstlisting}
PostalSTDIO : [(Integer Object) : 4]
PostalSTDIO : [(Integer Object) : 6]
\end{lstlisting}
\subsection{wrong sources}
\subsubsection{sntax error}
\begin{lstlisting}
    a=4
\end{lstlisting}
(we omit the ;)\\
result:
\begin{lstlisting}
Error executing code : syntax error : "" cannot begin an element of type<element suffixe>
\end{lstlisting}
\subsubsection{sntax error}
\begin{lstlisting}
m. = 4;   
\end{lstlisting}
(Correct sentence m.a = 4; for instance) \\
result:
\begin{lstlisting}
Error executing code : syntax error : expected "IDENTIFIER " but "=" found
\end{lstlisting}
\subsubsection{execution error b not bound}
\begin{lstlisting}
    a = 4 + b;
\end{lstlisting}
result:
\begin{lstlisting}
    Error executing code : trying to access b but not found
\end{lstlisting}
\subsubsection{execution error addition of Integer and Boolean}
\begin{lstlisting}
    a = 4 + true;
\end{lstlisting}
result:
\begin{lstlisting}
    Error executing code : Trying to add a non integer to an integer.
\end{lstlisting}
\end{document}
