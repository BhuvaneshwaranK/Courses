\documentclass{eplDoc}

\usepackage{placeins}

\newcommand{\docType}	{Assignment 2 : Graph Mining}
\newcommand{\docDate}	{18/05/2012}
\newcommand{\docAuthor}	{gr10 : Mulders Corentin, Pelsser Francois}
\newcommand{\courseCode}{SINF2275}
\newcommand{\courseName}{Data mining and decision making}
\usepackage{syntax}

\lstset{breaklines=true, breakatwhitespace=false}

\begin{document}
\maketitle
\newpage

\section{Implemented algorithms}

We chose to implement the algorithms from chapters 6.1, 8.1 and 8.2. So the following algorithms have been implemented : 33, 34, 47, 48, 49, 50. \\ 
 Here are the prototypes of our implementations, these can also be obtained by typing "help algoXX" in octave (where algoXX is the function name corresponding to the algorithm number XX)  :   

\subsection{Algorithm 33}
ALGO33(A, X, m, Y, lambda, mu) Performs a laplacian regularized least square for labeling the 
nodes of a weighted undirected graph G and integrating features availables on the nodes.
\begin{itemize}
	\item A is the adjacency matric representing G
  \item  X contains the features vectors as rows
  \item  m is the number of classes
  \item  Y a mxn matrix containing the m binary indicator vectors y(c) representing the classes as rows
  \item  lambda and mu are regularization parameters 
\end{itemize}
 Output : the membership matrix, each column representing a class.

\subsection{Algorithm 34}

ALGO34(A, K, m, Y, lambda, mu) Performs a laplacian regularized least square for labeling the 
nodes of a weighted undirected graph G and integrating features availables on the nodes.
\begin{itemize}
	\item  A is the adjacency matric representing G
  \item  K is the similarity matrix containing the similarities between nodes
  \item  m is the number of classes
  \item  Y a mxn matrix containing the m binary indicator vectors y(c) representing the classes as rows
  \item  lambda and mu are regularization parameters 
\end{itemize}
 Output : the membership matrix, each column representing a class.

\subsection{Algorithm 47}

ALGO47(A) Computes the core numbers of an unweighted undirected graph G 
\begin{itemize}
	\item A is the adjacency matrix representing G
\end{itemize}
 Output : A vector containing the core numbers associated to each node.



\subsection{Algorithm 48}
ALGO48(A, k) Computes the generalized k-core of an unweighted undirected graph G 
\begin{itemize}
	\item  A is the adjacency matrix representing G
   \item k is the order of the core
\end{itemize}
 Output : A vector containing the core numbers associated to each node. 


\subsection{Algorithm 49}
ALGO49(A, k) Computes the generalized k-core of an unweighted undirected graph G 
\begin{itemize}
	\item  A is the adjacency matrix representing G
   \item k is the order of the core
\end{itemize}
 Output : A vector containing the core numbers associated to each node. 


\subsection{Algorithm 50}

ALGO50(K) Computes a kernel ward hierarchical clustering of the nodes of a graph G
  
\begin{itemize}
	\item K is the similarity matrix associated with G
\end{itemize}
 Output : A matrix representing the dendogram of the hierarchical clustering, each 
 column represents a cluster and contains n elements $e_i$ with $e_i=0$ if the node $i$ is not in the cluster and 
 $e_i>0$ if it is in the cluster. \\ 
 A secondary output dendoinert gives for each cluster created the within-cluster inertia of the merge. 


\section{Implementation choices}

\subsection{Measure used for the algorithms 48 and 49}
In the algorithms 48 and 49 we use the degree of the nodes as measure for the computeMeasure subroutine. 

\subsection{Structure used to store the dendogram for algorithm 50}
In algorithm 50 we simply store the dendogram as a matrix in which each column represents a cluster. Initially it is an identity matrix of size n since the only clusters are the nodes themselves. Then each time we merge two clusters we add a column to this matrix with elements $e_i$ having a non-zero value only if the node $i$ belongs to the new cluster. 


\end{document}
