\documentclass{eplDoc}



\newcommand{\docType}	{Assignment 1}
\newcommand{\docDate}	{24/02/2012}
\newcommand{\docAuthor}	{gr16 : Mulders Corentin, Pelsser Francois}
\newcommand{\courseCode}{LINGI2365}
\newcommand{\courseName}{Constraint programming}
\usepackage{syntax}

\lstset{breaklines=true, breakatwhitespace=false}

\begin{document}
\maketitle
\newpage

\section{The rabbits and chicken problem}

We simply had to post two constraints to make sure that the total number of feet and the total number of heads are correct. 
\lstset{breaklines=true, breakatwhitespace=false}
\begin{lstlisting}
cp.post(rabbits*feetPerRabbit+chicken*feetPerChicken == feet);
cp.post(rabbits+chicken == heads);
\end{lstlisting}

The solution is that there are 28 rabbits and 44 chicken in the courtyard. 


\section{Nightmare in INGI}
\subsection{Lines added to the program}
\begin{lstlisting}

   // Exactly three people were in INGI at the time of the murder: Alice, Bob and Chuck
   cp.post((killer == alice) || (killer == bob) || killer == chuck);

   // The killer is taller than his victim
   cp.post( taller[killer,alice] ==1 );

     
   // Alice hates everybody except Bob
   ///!\ means Alice hates herself...
   forall(p in Persons : p != bob) cp.post(hates[alice,p] == 1);

   // Bob hates everyone that Alice hates
   forall(p in Persons) cp.post(hates[alice,p] != 1 || hates[bob,p] == 1);
   

   // Chuck hates no one that Alice hates
   forall(p in Persons) cp.post(hates[alice,p] != 1 || hates[chuck,p] != 1);

   //Nobody hates everyone
   forall(p in Persons) cp.post( (sum(b in Persons) hates[p,b]) < Persons.getSize());
\end{lstlisting}
\subsection{Solution}
The killer is Bob and there is no possible different killer. The code returned 8 solutions.   



\section{NQueens}
\subsection{Assign only values that are still in the domain}
\begin{lstlisting}
      forall(i in S : !q[i].bound())
        tryall<m>(p in S: q[i].memberOf(p))
          label(q[i], p);
\end{lstlisting}
\subsection{Assign variables with the smallest number of values in their domains first}
\begin{lstlisting}
      forall(i in S : !q[i].bound()) by (q[i].getSize())
        tryall<m>(p in S: q[i].memberOf(p)) 
          label(q[i], p);
\end{lstlisting}
\subsection{Assign queens in the even columns first}    
\begin{lstlisting}  
      forall(i in S : !q[i].bound()) by (i%2) 
        tryall<m>(p in S: q[i].memberOf(p)) 
          label(q[i], p);
\end{lstlisting}          
\subsection{Assign queens in the even columns first and try to assign values that are less present in the domains of other variables first}   
\begin{lstlisting}  
      forall(i in S : !q[i].bound()) by (i%2) 
        tryall<m>(p in S: q[i].memberOf(p)) by (sum(b in S) q[b].memberOf(p))
          label(q[i], p);
\end{lstlisting}          
\subsection{Assign queens from the sides to the center}     
\begin{lstlisting}
      forall(i in (1..4))
      {
        if(!q[i].bound()) 
          tryall<m>(p in S: q[i].memberOf(p))
            label(q[i], p);
        if(!q[n-i].bound()) 
          tryall<m>(p in S: q[n-i].memberOf(p))
            label(q[n-i], p);
       } 
\end{lstlisting}
\subsection{Split the domain of an unbound queen in 2}      
\begin{lstlisting}
        Solver<CP> cp = q[q.getLow()].getSolver();
        while (!bound(q)) {
          selectMin (i in q.getRange() : !q[i].bound()) (q[i].getSize()) {
            int mid = (q[i].getMin() + q[i].getMax()) / 2;
            try<cp> cp.post(q[i] <= mid); | cp.post(q[i] > mid);
          }
        }
        
      
\end{lstlisting}

\section{Baffling sequence}


\subsection{Formulation as a CSP problem}
A basic way of formulating this problem as a CSP problem would be with the following constraints : 
\begin{lstlisting}
forall(a in R) 
    cp.post((sum(r in R) (q[r]==a)) == q[a]);
\end{lstlisting}

\subsection{Two redundant constraints}
We can imagine the two following constraints on q for this problem. 
\begin{lstlisting}
forall(a in R : a != 0) 
    cp.post(q[a] <= n/a);
  
cp.post((sum(r in R) (q[r])) == n);
\end{lstlisting}

\subsection{How can the cardinality global constraint be used to solve the problem}
We can replace our basic constraint by : 
\begin{lstlisting}
cp.post(cardinality(q,q));
\end{lstlisting}


\subsection{Solutions}
To find the example solutions we used the basic formulation from 4.1 (model 1) then we used our redundant constraints and the cardinality global constraint (model 2). 

\subsubsection{Solution of size 100}
A example solution of size 100 found by our program is : \\ 

\begin{lstlisting}
[96,2,1,0,0,0,0,0,0,0,0,0,0,0,0,0,0,0,0,0,0,0,0,0,0,0,0,0,0,0,0,0,0,0,0,0,0,0, 0,0,0,0,0,0,0,0,0,0,0,0,0,0,0,0,0,0,0,0,0,0,0,0,0,0,0,0,0,0,0,0,0,0,0,0,0,0,0, 0,0,0,0,0,0,0,0,0,0,0,0,0,0,0,0,0,0,0,1,0,0,0]
\end{lstlisting}
\subsubsection{Solution of size 1000}
A example solution of size 1000 found with by model 2 in 11482 milliseconds (the process with model 1 was killed after 1min of not having found the solution) is : \\
\begin{lstlisting}
[996,2,1,0,0,0,0,0,0,0,0,0,0,0,0,0,0,0,0,0,0,0,0,0,0,0,0,0,0,0,0,0,0,0,0,0,0,0, 0,0,0,0,0,0,0,0,0,0,0,0,0,0,0,0,0,0,0,0,0,0,0,0,0,0,0,0,0,0,0,0,0,0,0,0,0,0,0,0, 0,0,0,0,0,0,0,0,0,0,0,0,0,0,0,0,0,0,0,0,0,0,0,0,0,0,0,0,0,0,0,0,0,0,0,0,0,0,0,0, 0,0,0,0,0,0,0,0,0,0,0,0,0,0,0,0,0,0,0,0,0,0,0,0,0,0,0,0,0,0,0,0,0,0,0,0,0,0,0,0, 0,0,0,0,0,0,0,0,0,0,0,0,0,0,0,0,0,0,0,0,0,0,0,0,0,0,0,0,0,0,0,0,0,0,0,0,0,0,0,0, 0,0,0,0,0,0,0,0,0,0,0,0,0,0,0,0,0,0,0,0,0,0,0,0,0,0,0,0,0,0,0,0,0,0,0,0,0,0,0,0, 0,0,0,0,0,0,0,0,0,0,0,0,0,0,0,0,0,0,0,0,0,0,0,0,0,0,0,0,0,0,0,0,0,0,0,0,0,0,0,0, 0,0,0,0,0,0,0,0,0,0,0,0,0,0,0,0,0,0,0,0,0,0,0,0,0,0,0,0,0,0,0,0,0,0,0,0,0,0,0,0, 0,0,0,0,0,0,0,0,0,0,0,0,0,0,0,0,0,0,0,0,0,0,0,0,0,0,0,0,0,0,0,0,0,0,0,0,0,0,0,0, 0,0,0,0,0,0,0,0,0,0,0,0,0,0,0,0,0,0,0,0,0,0,0,0,0,0,0,0,0,0,0,0,0,0,0,0,0,0,0,0, 0,0,0,0,0,0,0,0,0,0,0,0,0,0,0,0,0,0,0,0,0,0,0,0,0,0,0,0,0,0,0,0,0,0,0,0,0,0,0,0, 0,0,0,0,0,0,0,0,0,0,0,0,0,0,0,0,0,0,0,0,0,0,0,0,0,0,0,0,0,0,0,0,0,0,0,0,0,0,0,0, 0,0,0,0,0,0,0,0,0,0,0,0,0,0,0,0,0,0,0,0,0,0,0,0,0,0,0,0,0,0,0,0,0,0,0,0,0,0,0,0, 0,0,0,0,0,0,0,0,0,0,0,0,0,0,0,0,0,0,0,0,0,0,0,0,0,0,0,0,0,0,0,0,0,0,0,0,0,0,0,0, 0,0,0,0,0,0,0,0,0,0,0,0,0,0,0,0,0,0,0,0,0,0,0,0,0,0,0,0,0,0,0,0,0,0,0,0,0,0,0,0, 0,0,0,0,0,0,0,0,0,0,0,0,0,0,0,0,0,0,0,0,0,0,0,0,0,0,0,0,0,0,0,0,0,0,0,0,0,0,0,0, 0,0,0,0,0,0,0,0,0,0,0,0,0,0,0,0,0,0,0,0,0,0,0,0,0,0,0,0,0,0,0,0,0,0,0,0,0,0,0,0, 0,0,0,0,0,0,0,0,0,0,0,0,0,0,0,0,0,0,0,0,0,0,0,0,0,0,0,0,0,0,0,0,0,0,0,0,0,0,0,0, 0,0,0,0,0,0,0,0,0,0,0,0,0,0,0,0,0,0,0,0,0,0,0,0,0,0,0,0,0,0,0,0,0,0,0,0,0,0,0,0, 0,0,0,0,0,0,0,0,0,0,0,0,0,0,0,0,0,0,0,0,0,0,0,0,0,0,0,0,0,0,0,0,0,0,0,0,0,0,0,0, 0,0,0,0,0,0,0,0,0,0,0,0,0,0,0,0,0,0,0,0,0,0,0,0,0,0,0,0,0,0,0,0,0,0,0,0,0,0,0,0, 0,0,0,0,0,0,0,0,0,0,0,0,0,0,0,0,0,0,0,0,0,0,0,0,0,0,0,0,0,0,0,0,0,0,0,0,0,0,0,0, 0,0,0,0,0,0,0,0,0,0,0,0,0,0,0,0,0,0,0,0,0,0,0,0,0,0,0,0,0,0,0,0,0,0,0,0,0,0,0,0, 0,0,0,0,0,0,0,0,0,0,0,0,0,0,0,0,0,0,0,0,0,0,0,0,0,0,0,0,0,0,0,0,0,0,0,0,0,0,0,0, 0,0,0,0,0,0,0,0,0,0,0,0,0,0,0,0,0,0,0,0,0,0,0,0,0,0,0,0,0,0,0,0,0,0,0,0,0,0,1,0, 0,0]
\end{lstlisting}
\end{document}
